\documentclass[11pt,a4paper]{article}
%PKG
\usepackage[utf8]{inputenc}
\usepackage[english]{babel}
\usepackage{amsmath}
\usepackage{amsfonts}
\usepackage{amssymb}
\usepackage{hyperref}
\usepackage{textcomp}

\begin{document}
\pagestyle{plain}
\author{Axel Jeanne \and Monica Arias Rivera}
\title{Cooperative supervised mobile manipulator}
\maketitle
\begin{abstract}
Robotic systems grew in complexity and diversity due to the increasing complexion of the task
they are performing. As complexity increase, decoupling element into smaller simpler part is
regularly done. As a part if this decoupling trend, cooperative robotics grant an even 
larger change: using multiple robot to perform a single, yet complex, task.

This report present the work performed during the Supervised Mobile Manipulator (SMM) project.
The project consists in making a cooperative robotic system providing "smart" behaviour to
make the tele-operation easier to the user while keeping effectiveness of the controlled robot 
unchanged.

For this project we used different hardware and software elements that we will introduce,
then we will present different cooperative behaviours and will expose their strengths and weaknesses; finally we will present the chosen system and expose its features and the results we obtained.
\end{abstract}
\tableofcontents

\section{Theoretical content}
\subsection{Advantages of cooperative system}
In most cases, the operator performing a task remotely only have a single point of view: the
one of the embedded camera. In multiple different systems, it has been shown that this can
lead to inconvenient situations where the operator is performing a task with a sub-optimal
vision of the situation.

One solution to this problem could be to add an animate vision system\cite{Ballard1991}
but this kind of solution increase vastly complexity of the system since all the vision system
has to be actuated. Another solution could be to increase the number of camera on the robot.
This solution increase the robot payload and is actually not very convenient for the user
since more video streams has now to be monitored while performing the task. The solution we chose is to add an external robot to provide a "supervisor" vision of the acting robot. This solution adds many different  advantages:

\begin{enumerate}
\item The supervisor can provide vision from different angles while the action robot is static
\item The supervisor can scout ahead of the action robot to ensure safety
\item The supervisor behaviours can be automated to unload the operator's attention
\end{enumerate}


\subsection{Problem statement}
The Supervise Mobile Manipulator Project (SMM) goal is to develop a cooperative robotics
system which can be used in hazardous areas. It uses two platforms: 

\begin{itemize}
\item An action platform which perform the physical operation
\item An supervisor platform which keep an overview of the situation
\end{itemize}

To deliver a fully applicable solution, a GUI (Graphical User Interface) had to be provided in
order to see what is the system doing in real time.


\subsection{Tool used}
\subsubsection{Git}
To deliver a complete solution, a lot of software development was required. In order to keep
the code maintained and keep track of changes in code; we used \href{https://git-scm.com/}{git} as a version control system
 (VCS).
\subsubsection{ROS} \label{ROS}
ROS stands for Robotic Operating System; ROS is not technically a OS but provides many abstraction for network communications among robots. As we require cooperation among our 
system, ROS was a very useful tool.
\subsubsection{Qt}
Qt is a well known GUI library. Is is used a large variety of applications from mobile phone interfaces to advanced customized programs. Qt is so popular that a module for ROS was
created called rqt (ROS Qt).

\subsection{Mobile base}
The mobile base used is a TurtleBot. TurtleBot is a low-cost, personal robot kit with open-source software provided by different partners mostly for educational activities.
Our mobile base was equipped with a Microsoft\textcopyright Kinect\texttrademark a cheap camera which is able to compute the depth of an image easily.

\subsection{Quad-rotor}
The quad-rotor used is a Parrot\textcopyright AR Drone 2\texttrademark it is a 4 propellers
drone which has a ROS compatible driver. It is also equipped with two cameras: one in front
and another in the bottom. The bottom camera can be used for visual tracking and the drone
has an on-board tracking system, allowing a decent tracking without latency issues.
Additionally, some ROS libraries have been created to be compatible with the drone; we used
one of them called \href{"http://wiki.ros.org/tum_ardrone"}{"tum\_ardrone"} in parallel with
the driver package: \href{"https://github.com/AutonomyLab/ardrone_autonomy"}
{"ardrone\_autonomy"}.

\subsection{Cooperative behaviour}
As the two robots have to cooperate to perform their common goal, it was required to implement
cooperative behaviours on the system.

\section{Work achieved}
\subsection{GUI programming}
\subsection{Robots interactions}
\subsection{Simulator implementation}
\subsection{Testing and first results}

\section{Result analysis and conclusions}
\subsection{Final results analysis}
\subsection{Improvement and future work}
\subsection{Conclusion}


\bibliography{bib}
\bibliographystyle{abbrv}
\end{document}